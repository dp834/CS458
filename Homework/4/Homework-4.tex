\documentclass{article}

\usepackage[margin=.75in]{geometry}
\usepackage{amsmath}
\usepackage{amssymb}
\usepackage{tabu}
\usepackage[shortlabels]{enumitem}

\author{Damien Prieur}
\title{Homework 4 \\ CS 458}
\date{}

\begin{document}

\maketitle

\section*{Problem 1}
Consider the following integer linear program that generalizes vertex cover
\newline
\begin{table}[h]
\centering
$\begin{tabu}{c c c}
\text{minimize}   & c_1x_1 + ... + c_mx_m       &          \\
\text{subject to} &a_{i,1}x_1 + ... + a_{i,m}x_m& \geq b_i \quad \forall i \in \{1,2,...,n\} \\
                  & x_j  & \in \{0,1\} \quad \forall j \in \{1,2,...,n\}
\end{tabu}$
\end{table}

We will assume that the parameters satisfy the following conditions
\begin{itemize}
\item $a_{i,j},b_i,c_j$ are all non-negative integers
\item $b_i,c_j > 0$
\item A feasible solution exists, namely $\Sigma_j a_{ij} \geq b_i$
\item and let $f = \max_{i=1,...,n}(\Sigma_{j=1}^m a_{i,j})$.
\end{itemize}

\begin{enumerate}[a.]
\item State how the integer linear program generalizes vertex cover by defining $a_{i,j},c_j,b_i$ and $f$ in terms of a specific instance of vertex cover.
\begin{itemize}
\item $x_i \quad i \in \{1,...,m\}$ represents an vertex, binary variable either selected or not
\item $a_{i,j} \quad i \in \{1,...,n\}, j \in \{1,...,m\}$ represents the 'reward' of edge $i$ being connected by vertex $j$
\item $b_{i} \quad i \in \{1,...,n\}$ represents the minimum total 'reward' a edge must satisfy.
\item $c_{i} \quad i \in \{1,...,m\}$ represents the cost of picking vertex $m$
\item $f$ represents the maximum total 'reward' edge possible.
\end{itemize}

\item Give a polynomial-time approximation algorithm for any problem captured by this general integer linear programming formulation with approximation ratio \underline{less than or equal to} $f$ (you need to provide an upper bound).
In other words given any linear program that satisfies the conditions outlined above give an algorithm that has approximation ratio $f$.
\newline
Given that we have a LP integer solution we can see how the relaxed version of the problem does when rounded.
We begin with the same problem as is given except the final constraint is changed from $x_j \in \{0,1\}$ to $ 0 \leq x_j \leq 1$.
Once we have a fractional solution we need to determine how to round.
Looking at the first constraint, since we are talking about an edge only 2 of these variables $a_{i,j}$ will contribute to the sum.
If we round both up since we don't know how much the $b_i$ term will effect other edges picked we can't accidentally undercover.
In the worst case we will increase an edge by at most $f$ based on it's definition.
$$\Sigma_{i=1}^m c_ix_i^{INT} \leq \Sigma_{i=1}^m c_i(f x_i) = f\Sigma_{i=1}^m c_ix_i = f \text{ Min GVC} $$
I don't really see how $f$ connects to the objective function in any direct way, so this can't be correct.
\end{enumerate}


\newpage
\section*{Problem 2}
Consider the \emph{maximum cardinality matching problem on general graphs}:
We seek to find a maximum cardinality subset of edges $M \subseteq E$ such that no edges $e, e' \in M$ share an endpoint.
The integer linear programming formulation is the following:

\begin{table}[h]
\centering
$\begin{tabu}{c c c c}
\text{maximize}   & \sum_{e \in E} x_e & & \\
\text{subject to} & \sum_{e\text{ incident to } v} x_e & \leq 1 & \forall v \in V \\
                  & x_e   & \in \{0,1\} & \forall e \in E
\end{tabu}$
\end{table}
Write its linear programming relaxation and show that the integrality gap is \underline{greater than or equal to} $\frac{3}{2}$ (you need to provide a lower bound)
\newline
Relaxed version just removes the integrality restriction and allows for the whole range from $0$ to $1$:
\begin{table}[h]
\centering
$\begin{tabu}{c c c c}
\text{maximize}   & \sum_{e \in E} x_e & & \\
\text{subject to} & \sum_{e\text{ incident to } v} x_e & \leq 1 & \forall v \in V \\
                  & x_e   & \in [0,1] & \forall e \in E
\end{tabu}$
\end{table}
We can show that there is always a fractional feasible solution of $\frac{|e|}{|V|-1}$.
\newline
Set $x_e = \frac{1}{|V_{\text{sub-graph}}-1|} \quad \forall e \in E \text{for each sub-graph }$.
Look at each sub-graph in the graph and count the number of vertices.
For each sub-graph assign the edges in that sub-graph the value $\frac{1}{|V_{sub}-1}$
All constraints are satisfied since we would be adding at most $|V|-1$ edges together which sums to exactly $1$ for each subgraph.
In the case where we have $n$ complete sub-graphs with $3$ vertices the optimal integer solution is to select one edge from each so $n$ total edges.
Each sub-graph has $3$ vertices and $3$ edges so the fractional solution would be $\frac{3n}{(3-1)} = \frac{3n}{2}$
The integrality gap is $\frac{\text{FRAC}}{\text{INT}}$ so:
$$ \lim_{n \to \infty} \frac{\frac{3n}{2}}{n} = \frac{3}{2} $$


\newpage
\section*{Problem 3}
Given an undirected graph $G=(V,E)$, suppose we want to remove a \textbf{minimum number} of edges $F \subseteq E$ such that the remaining graph $G' = (V,E-F)$ contains no triangle.
(I.e., there are no three vertices $u,v,w$ such that edges $(u,v),(v,w),(w,u)$ are all in $G'$.)
\begin{enumerate}[a.]
\item Give a polynomial-time approximation algorithm to solve this problem with approximation ratio \underline{less than or equal to} 3 (you need to provide an upper bound).
\newline
This algorithm will be greedy by searching for triangles in the graph and removing edges from the triangle.
Just like the example in class the algorithm will remove every edge in the triangle as that improved the previous algorithm's performance and made for easier analysis.
\newline
For each triple of vertices of which there are ${n \choose 3} \in O(|V|^3)$ if any have edges that form a triangle add all $3$ edges to the set $F$.
Remove these edges from $G$ as well, and repeat this process until there are not more triangles.
At most this will run $|E|/3$ times as each time $3$ edges would be removed.
Therefore the runtime is $O(n^3m)$ which is polynomial.
\newline
For the upper bound for the approximation we can look at the resulting set $F$.
Each triangle in the set $F$ will be disjoint as we remove them a triangle at a time.
The minimum set $F$ must contain at least one edge from each triangle we removed or else a triangle would still exist in the original graph.
For each triangle that we generated in $F^{OPT}$ the algorithm's $F$ will have removed $2$ extra edges per triangle.
This give us an approximation ratio of $\frac{2+1}{1} = 3$


\item Provide a lower bound that matches the upper bound you proved in part (a).
\newline
If we start with the set $F^{OPT}$ we can build a set of triangles that could be removed from our algorithm.
First we need to see that not two edges in $F^{OPT}$ are part of the same original triangle in $G$ or it could remove it from the final set $F$.
If our algorithm were to identify each of these triangles we would remove the entire triangle for each edge in $F^{OPT}$.
So just as before we are always over counting by $2$ edges giving the same ratio of $\frac{2+1}{1} = 3$
\end{enumerate}

\end{document}

