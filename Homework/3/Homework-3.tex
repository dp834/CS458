\documentclass{article}

\usepackage[margin=.75in]{geometry}
\usepackage{amsmath}
\usepackage{amssymb}
\usepackage{tabu}
\usepackage[shortlabels]{enumitem}

\author{Damien Prieur}
\title{Homework 3 \\ CS 458}
\date{}

\begin{document}

\maketitle

\section*{Problem 1}
In class, Linear Programming was formulated in the following way:
\newline
Inputs:
\begin{itemize}
\item $n$ decision variables $x_1, ... , x_n$.
\item $m$ linear inequality constraints. Constraint $j$ is give by $a_{1j}x_1 + ... + a_{nj}x_n \leq b_j$.
\item Linear objective function $c_1x_1 + ... c_nx_n$.
\end{itemize}
Goal: Find values $x_1^*, ... , X_n^*$ for decision variables such that
\begin{itemize}
\item Constraints are satisfied.
\item $x_1^*, ... ,x_n^*$ is maximized
\end{itemize}

\indent This statement of the input, with a maximization objective, "less-then" constraints, and non-negative variables, is often referred to as "standard form" for linear programs.
In this problem you'll show that linear programming can handle a few other things which on face may not even seem "linear" at all.
In each part, you'll be given a mathematical program, and you job is to write an equivalent linear program in canonical form.
You don't need to offer any proof, but you should briefly explain why the two programs are equivalent, and how to go translate between solutions to the two programs.
\newline
[Hint, for all three parts: define extra variables.]

\begin{enumerate}[1.]
\item \emph{Unconstrained Variables}
\newline
\begin{table}[h]
\centering
$\begin{tabu}{c c c}
\text{maximize}   & c_1x_1+ ... + c_nx_n       &          \\
\text{subject to} & a_{11}x_1+ ... + a_{1n}x_n & \leq b_1 \\
                  & a_{21}x_1+ ... + a_{2n}x_n & \leq b_2 \\
                  &            ...             &          \\
                  & a_{m1}x_1+ ... + a_{mn}x_n & \leq b_m
\end{tabu}$
\end{table}
[Note the absence of nonegativity constraints.]
\newline
\newline
We can split each variable $x_i$ into a new pair of variables $x_{ip}, x_{in}$ to be substituted using the following.
$x_i = x_{ip} - x_{in}$.
This allows for $x_i$ to reach any number while being represented by only positive values.
The new linear program would now look like this:
\begin{table}[h]
\centering
$\begin{tabu}{c c c}
\text{maximize}   & c_1(x_{1p}-x_{1n}) + ... + c_n(x_{np}-x_{nn})      &          \\
\text{subject to} & a_{11}(x_{1p}-x_{1n})+ ... + a_{1n}(x_{np}-x_{nn}) & \leq b_1 \\
                  & a_{21}(x_{1p}-x_{1n})+ ... + a_{2n}(x_{np}-x_{nn}) & \leq b_2 \\
                  &            ...                                     &          \\
                  & a_{m1}(x_{1p}-x_{1n})+ ... + a_{mn}(x_{np}-x_{nn}) & \leq b_m \\
                  & x_{1p}, x_{1n}, ... , x_{1p}, x_{1n}               & \geq 0 \phantom{-}
\end{tabu}$
\end{table}
\newpage

\item \emph{Maximin Objective:}
\begin{table}[h]
\centering
$\begin{tabu}{c c c}
\text{maximize}   & \min(x_1, ... ,x_n)        &          \\
\text{subject to} & a_{11}x_1+ ... + a_{1n}x_n & \leq b_1 \\
                  & a_{21}x_1+ ... + a_{2n}x_n & \leq b_2 \\
                  &            ...             &          \\
                  & a_{m1}x_1+ ... + a_{mn}x_n & \leq b_m \\
                  & x_1, ... , x_n             & \geq 0
\end{tabu}$
\end{table}
\newline
Replace the objective function with a new variable $X$.
All we need is for $X$ to be equal to the smallest $x_i$, so we must add a constraint for each $x_i$.
$$X \leq x_i \quad \forall i \in {1,..,n}$$
This forces $X$ to equal the smallest $x_i$ as it is not being constrained by anything else.
The new linear program would now look like this:
\begin{table}[h]
\centering
$\begin{tabu}{c c c}
\text{maximize}   &            X               &          \\
\text{subject to} & a_{11}x_1+ ... + a_{1n}x_n & \leq b_1 \\
                  & a_{21}x_1+ ... + a_{2n}x_n & \leq b_2 \\
                  &            ...             &          \\
                  & a_{m1}x_1+ ... + a_{mn}x_n & \leq b_m \\
                  &            X \leq x_i      & \forall i \in {1,n} \\
                  & x_1, ... , x_n             & \geq 0
\end{tabu}$
\end{table}

\item \emph{Absolute Value}
\newline
\begin{table}[h!]
\centering
$\begin{tabu}{c c c}
\text{minimize}   & |c_1x_1 + ... + c_nx_n|    &          \\
\text{subject to} & a_{11}x_1+ ... + a_{1n}x_n & \leq b_1 \\
                  & a_{21}x_1+ ... + a_{2n}x_n & \leq b_2 \\
                  &            ...             &          \\
                  & a_{m1}x_1+ ... + a_{mn}x_n & \leq b_m \\
                  & x_1, ... , x_n             & \geq 0
\end{tabu}$
\end{table}
\newline
We need to represent the absolute value in terms of its two components.
Instead of looking at the absolute value as a piecewise it is easier to use it as $|x| = max{-x,x}$ for the situation above.
We can model this by adding a new variable $X$ and forcing it to be larger than $c \cdot x$ and $- (c \cdot x)$.
Which we can add to our constraint list with:
$$ c_1x_1 + ... + c_nx_n \leq X \quad \text{and} \quad -(c_1x_1 + ... + c_nx_n) \leq X $$
Written in Canonical form we have
$$ c_1x_1 + ... + c_nx_n - X \leq 0 \quad \text{and} \quad -(c_1x_1 + ... + c_nx_n) - X \leq 0 $$
To change the minimization to a maximization we can use the fact that minimizing $f$ is the same as maximizing $-f$.
The new linear program would now look like this:
\begin{table}[h!]
\centering
$\begin{tabu}{c c c}
\text{maximize}   &            -X              &          \\
\text{subject to} & a_{11}x_1+ ... + a_{1n}x_n & \leq b_1 \\
                  & a_{21}x_1+ ... + a_{2n}x_n & \leq b_2 \\
                  &            ...             &          \\
                  & a_{m1}x_1+ ... + a_{mn}x_n & \leq b_m \\
                  & c_1x_1 + ... + c_nx_n - X  & \leq 0   \\
                  & -(c_1x_1 + ... + c_nx_n)-X & \leq 0   \\
                  & x_1, ... , x_n             & \geq 0
\end{tabu}$
\end{table}

\end{enumerate}

\newpage
\section*{Problem 2}
In this part, you are given several computational problems.
Your task is to formulate each one as an integer linear program.
You don't need to prove that the formulation is correct this time, but you should give an interpretation of the objective function and each constraint.
Your formulations do not need to be in canonical form.

\begin{enumerate}[a.]
\item \emph{MAX 3-SAT:} Given a boolean formula $\Phi(x_1,...,x_n)$ in 3-conjunctive normal form, find a truth assignment for $x_1, ... , x_n$ which maximizes the number of clauses satisfied.
For clause $C_j$, let $y_i \in \{0,1\}$ be a variable indicating whether or not $C_j$ is satisfied, and let $x_i \in \{0,1\}$ denote the truth value for the boolean variable $x_i$.
Using $y_i$ and $x_i$ as your variables, formulate this problem as an integer program.
\newline
\newline
Since we want to maximize the number of clauses we want to maximize over $\sum_i y_i$.
Constraints:
\begin{itemize}
\item $\sum_{i\in C_j} x_i \geq y_j \quad \forall C_j \in C$
\newline
For each clause we cannot let $y_i$ be 1 unless ANY of the 3 variables evaluates to true.
I'm not sure if I wrote it down correctly though.
\end{itemize}
This gives us
\begin{table}[h]
\centering
$\begin{tabu}{c c c c}
\text{maximize}   & \sum_i y_i & & \\
\text{subject to} & \sum_{i\in C_j} x_i & \geq y_j & \forall C_j \in C \\
                  & y_i   & \in \{0,1\} & \forall i \in C
\end{tabu}$
\end{table}

\item \emph{Fair k-Suppliers.} Given a set of customers $D$, a set of suppliers $F$, distances $d_{ij}$ between each client $j \in D$ and supplier $i \in F$, and a positive integer $k$, the goal is to choose a subset of $k$ suppliers $S \subseteq F$ so as to minimize the total distance the customers have to travel to their closest selected supplier, i.e. $\sum_{j \in D} d(j,S)$.
For each supplier $i \in F$ let $y_i \in \{0,1\}$ indicate whether we choose supplier $i$.
For each client $j \in D$ and each supplier $i \in F$, let $x_{i,j}$ indicate whether client $j$ is "assigned" $j$ to supplier $i$ (i.e. if $i$ is $j$'s closest supplier in $S$).
Using these $y_i$ and $x_{i,j}$ as your variables, formulate this problem as an integer program.
\newline
\newline
Our goal is $\min \sum_{i \in F}\sum_{j \in D} d_{i,j} \cdot x_{ij}$ which is the sum of the distances of chosen pairs.
\newline
Constraints:
\begin{itemize}
\item $ \sum_i y_i \leq k $
\newline
We can not choose more than $k$ suppliers
\item $ \sum_i x_{i,j} \geq 1 \quad \forall j \in D $
\newline
Each client should get supplies from at least one supplier. No more than one will be selected as due to the minimization and distances being positive.

\item $ x_{i,j} \leq y_i \quad \forall i \in F, j \in D$
\newline
Assignments of suppliers to clients can only be made if the supplier is picked.

\end{itemize}

This gives us
\begin{table}[h]
\centering
$\begin{tabu}{c c c c}
\text{minimize}   & \sum_{i \in F}\sum_{j \in D} d_{i,j} \cdot x_{ij} & &   \\
\text{subject to} & \sum_i y_i                 & \leq k   & \\
                  & \sum_i x_{i,j}             & \geq 1   & \forall j \in D \\
                  & x_{i,j}                    & \leq y_i & \forall i \in F, j \in D\\
                  & y_i                     & \in \{0,1\} & \forall i \in F \\
                  & x_{i,j}                 & \in \{0,1\} & \forall i \in F, j \in D
\end{tabu}$
\end{table}

\item \emph{Weighted Independent Set.} Given an undirected graph $G = (V,E)$ with vertex weights $w_v \geq 0$, the goal is to find an independent set of maximum weight (independent set is a set of vertices such that there is no edge between any pair of vertices in the set).
For each vertex $v$, let $x_v \in \{0,1\}$ indicate whether $v$ is in the independent set.
Using these $x_v$ as your variables, formulate the maximum-weight independent set problem as an integer program.
\newline
\newline
Let $e_{i,j} \in \{0,1\}$ be the representation of an edge between two vertices in the graph with a $1$ indicating an edge.
Our goal is to maximize $\sum_{v\in V} w_v \cdot x_v$ which is the sum of selected vertices.
Constraints:
\begin{itemize}
\item $ x_{i} + \sum_{v\in V \setminus {i}} x_v \cdot e_{i,v} \leq 1 \quad \forall i \in V$
\newline
If a vertex is selected ensure that no adjacent vertex is also selected.
\end{itemize}
This is the only constraint for feasible solutions.
This gives us:
\begin{table}[h]
\centering
$\begin{tabu}{c c c c}
\text{maximize}   & \sum_{v \in V}w_v\cdot x_v & &   \\
\text{subject to} & x_{i} + \sum_{v\in V \setminus {i}} x_v \cdot e_{i,v} & \leq 1 & \forall i \in V\\
                  & x_v                     & \in \{0,1\} & \\
\end{tabu}$
\end{table}

\item \emph{Load Balancing.} Given set $J$ of $n$ jobs and a set $M$ of $m$ machines, where job $j$ requires $p_j$ units of consecutive processing time and can only be assigned to a machine from the set $M_j \subseteq M$, the goal is to assign each job $j$ to some machine in $i \in M_j$ such that we minimize the maximum load on any machine (i.e the \emph{makespan}).
(If $J_i \subseteq J$ denotes the job is assigned to a machine $i \in M$ in a feasible assignment, and using $L_i = \sum_{j \in J_i}p_j$ to denote its resulting load, we seek to minimize $\max_i L_i$). Let $x_{i,j}$ indicate whether job $j$ is assigned to machine $i \in M_j$.
Also let $L$ denote the maximum load.
Using these as your variables, formulate the load balancing problem as an integer program.
\newline
\newline
Our objective was given in the problem statement $\min \max_i L_i$.
\begin{itemize}
\item $\sum_{i \in M_j} x_{i,j} = 1 \quad \forall j \in J$
\newline
A job must be assigned once.
\item $L_i = \sum_{j\in J_i}p_j \cdot x_{i,j} \quad \forall i \in M$
\newline
Just sets the load for a machine so we can use it elsewhere.
\end{itemize}
\end{enumerate}
This gives us:
\begin{table}[h]
\centering
$\begin{tabu}{c c c c}
\text{maximize}   & \min \max_i L_i & &   \\
\text{subject to} & \sum_{i \in M_j} x_{i,j} & = 1 & \forall j \in J\\
                  & x_{i,j}                 & \in \{0,1\} & \forall i,j \in J,M \\
\end{tabu}$
\end{table}


\newpage
\section*{Problem 3}
In class we show how to model maximum flow using variable $f_{u,v}$ for every $u,v \in V$. Additionally we showed how to model the minimum $s - t$ cut using variables $x_{u,v}$ and $y_v$ for every vertex in $u,v \in V$.
\newline
\indent In this problem you will give an alternative linear program using paths.
In all of your answers you have to provide a justification why the linear program correctly describes the problem you are given.

\begin{enumerate}[a.]
\item Let $\mathcal{P}$ denote the set of $s-t$ paths in $G$.
For each path $P \in \mathcal{P}$, let $x_P$ denote the flow on $P$.
Using these $x_P$ as your variables, formulate the maximum flow problem as a linear program.
\newline
\newline
The flow through the whole graph is going to be given by $\sum_{p \in \mathcal{P}} x_p$.
Constraints:
\begin{itemize}
\item $x_p \leq c(e) \quad \forall p \in \mathcal{P}, \forall e \in p$
\newline
The flow through a path must be smaller than every capacity on the path.
\item $x_p \geq 0$
\newline
The flow must be positive.
\end{itemize}
This gives us:
\begin{table}[h]
\centering
$\begin{tabu}{c c c c}
\text{maximize}   & \sum_{p \in \mathcal{P}} x_p & & \\
\text{subject to} & x_p & \leq c(e) & \forall p \in \mathcal{P}, \forall e \in p \\
                  & x_{p}                 & \geq 0 & \forall p \in \mathcal{P}\\
\end{tabu}$
\end{table}

\item For each edge $(u,v) \in E$, let $y_{uv} \in \{0,1\}$ indicate whether edge $(u,v)$ crosses the cut (that is, $y_{u,v} = 1$ if and only if $u \in S, v \in T$.)
Using these $y_{uv}$ as your variables, formulate the minimum cut problem as an (integer) linear program.
You are not allowed to use any other variables.
\newline
\newline
Our goal is to $\min \sum_{(u,v) \in E} c(u,v) \cdot y_{uv}$.
In other words we want to minimize the sum of capacities of edges that are being cut.
Constraints:
\begin{itemize}
\item $\sum_{(u,v) \in p} y_{uv}  = 1 \quad \forall p \in \mathcal{P}$
\newline
There can only be one edge cut per path.
\end{itemize}
This gives us:
\begin{table}[h]
\centering
$\begin{tabu}{c c c c}
\text{minimize}   & \sum_{(u,v) \in E} c(u,v) \cdot y_{uv}& & \\
\text{subject to} & \sum_{(u,v) \in p} y_{uv} &  = 1 & \forall p \in \mathcal{P} \\
                  & y_{uv}             & \in \{0,1\} & \forall (u,v) \in E \\
\end{tabu}$
\end{table}

\item Give the linear programming relaxation of the integer linear program you defined in part b.
Show that it is the dual of the linear program you defined in part a.
\newline
\begin{table}[h!]
\centering
$\begin{tabu}{c c c c}
\text{minimize}   & \sum_{(u,v) \in E} c(u,v) \cdot y_{uv}& & \\
\text{subject to} & \sum_{(u,v) \in p} y_{uv} &  = 1 & \forall p \in \mathcal{P} \\
                  & y_{uv}             & \geq 0  & \forall (u,v) \in E \\
                  & y_{uv}             & \leq 1  & \forall (u,v) \in E \\
\end{tabu}$
\end{table}
\newline
We can see that the constraints from part a are equal to the objective from part b.
And vice versa, one is a minimization while the other is maximization, and the constraints are just transposes of each other.

\item Assuming $c(u,v)$ are integers, show that the integrality gap of both linear programs is defined in part (a) and (c) is 1.
You can use any existing result you may know.
\newline
\newline
When pivoting on simplex variables will take as much slack as possible.
Since all elements that are pivotable have a coefficient of 1 when pivoting we are just adding subtracting and multiplying integers which all stay within the integers.
In our case that will always be some combination of capacities being added or subtracted together.
Since all capacities are integers then their sums and differences will also be integers.
Since we aren't restricting the solution space to just integers and we still will get an integer then the integrality gap is going to be 1.

\item Show that the integrality gap of the linear programming relaxation you define in part c. is still 1 even for arbitrary capacities $c(u,v) \in \mathbb{R}$ and $c(u,v) \geq 0$.
\newline
\newline
This is a consequence of the pivot operation never using the capacity value for filling in new entries.
All other entries in the table are 1.
Therefore basic variables will always be some additions and subtractions of 1 and 0 but still bounded by the constraint $0 \leq y_{uv} \leq 1$.
I know this is an incomplete explanation, but I'm not sure how to explain it very clearly or from a different angle.

\end{enumerate}

\end{document}

